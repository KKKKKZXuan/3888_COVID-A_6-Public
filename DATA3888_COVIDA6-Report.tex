\documentclass[letterpaper,9pt,twocolumn,twoside,]{pinp}

%% Some pieces required from the pandoc template
\providecommand{\tightlist}{%
  \setlength{\itemsep}{0pt}\setlength{\parskip}{0pt}}

% Use the lineno option to display guide line numbers if required.
% Note that the use of elements such as single-column equations
% may affect the guide line number alignment.

\usepackage[T1]{fontenc}
\usepackage[utf8]{inputenc}

% pinp change: the geometry package layout settings need to be set here, not in pinp.cls
\geometry{layoutsize={0.95588\paperwidth,0.98864\paperheight},%
  layouthoffset=0.02206\paperwidth, layoutvoffset=0.00568\paperheight}

\definecolor{pinpblue}{HTML}{185FAF}  % imagecolorpicker on blue for new R logo
\definecolor{pnasbluetext}{RGB}{101,0,0} %



\title{Factors Leading Severe COVID Symptoms and Death}

\author[true]{Changwa Yin}
\author[p]{Pat Thepyasuwan}
\author[c]{Zhanqiu Chen}
\author[m]{Zhekai Mao}
\author[d]{Zhenyao Dou}

  \affil[s]{The University of Sydney, NSW, 2008}

\setcounter{secnumdepth}{0}

% Please give the surname of the lead author for the running footer
\leadauthor{C.Yin, P.Thepyasuwan, Z.Chen, Z.Mao, Z.Dou}

% Keywords are not mandatory, but authors are strongly encouraged to provide them. If provided, please include two to five keywords, separated by the pipe symbol, e.g:
 \keywords{  covid |  maternal health  }  

\begin{abstract}
Factors which have been researched to determine whether it has an effect
on the severity of a COVID-19 case includes each country's ICU admission
rates, each country's Population Density, patients with underlying
diseases in each country, Mortality, as well as the patient's
vaccination status as well as whether the patient of interest is
currently residing in a developed or a developing country. Users can
select which factors are relevant to themselves in order to discover
relevant information.
\end{abstract}

\dates{This version was compiled on \today} 


% initially we use doi so keep for backwards compatibility
% new name is doi_footer
\doifooter{\url{https://github.sydney.edu.au/zmao0489/3888_COVID-A_6}}

\pinpfootercontents{owid/covid-19-data}

\begin{document}

% Optional adjustment to line up main text (after abstract) of first page with line numbers, when using both lineno and twocolumn options.
% You should only change this length when you've finalised the article contents.
\verticaladjustment{-2pt}

\maketitle
\thispagestyle{firststyle}
\ifthenelse{\boolean{shortarticle}}{\ifthenelse{\boolean{singlecolumn}}{\abscontentformatted}{\abscontent}}{}

% If your first paragraph (i.e. with the \dropcap) contains a list environment (quote, quotation, theorem, definition, enumerate, itemize...), the line after the list may have some extra indentation. If this is the case, add \parshape=0 to the end of the list environment.


\hypertarget{introduction}{%
\section{Introduction}\label{introduction}}

This report describes the questionnaires answered on ED by students
studying DATA2X02 in the second semester of 2021 at the University of
Sydney. The questionnaire contains data on the number of COVID tests,
current learning and living conditions, personal attributes, and hopes
for the future. After completing the data cleaning and verifying with
various data tests, we can draw the following conclusions.

\hypertarget{data-cleaning}{%
\section{Data Cleaning}\label{data-cleaning}}

First, convert the data from the given CSV file into a data frame in R.

%\showmatmethods


\bibliography{pinp}
\bibliographystyle{jss}



\end{document}
